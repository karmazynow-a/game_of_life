
%----------------------------------------------------------------------------------------
%	PREAMBUŁA
%----------------------------------------------------------------------------------------

\documentclass[12pt]{article}
\usepackage[polish]{babel}
\usepackage{polski}
\usepackage[utf8]{inputenc}
%\usepackage[T1]{fontenc}
\usepackage{amsmath}
\usepackage{graphicx}
\usepackage{fancyhdr}
\usepackage{float}
\usepackage{graphicx}
\usepackage{hyperref}
\usepackage{verbatim}

\usepackage{color} %red, green, blue, yellow, cyan, magenta, black, white
\definecolor{mygreen}{RGB}{28,172,0} % color values Red, Green, Blue
\definecolor{mylilas}{RGB}{170,55,241}


\title{Sprawozdanie}
\author{Aleksandra Poręba}

\graphicspath{{static/}{../images/}} 

\makeatletter
\let\thetitle\@title
\let\theauthor\@author
\let\thedate\@date
\makeatother


%----------------------------------------------------------------------------------------
%	STRONA TYTUŁOWA
%----------------------------------------------------------------------------------------
\begin{document}
\begin{center}
\textsc{\normalsize Wydział Fizyki i Informatyki Stosowanej}\\[2.0cm] 
\includegraphics[scale = 1]{logo.png}\\[1cm] 
%\textsc{\Large Modelowanie Procesów Fizycznych}\\[0.4cm] 


\textsc{\Large Sprawozdanie}\\[0.4cm]
{ \huge \bfseries \LARGE{Projekt 1: Gra w życie} }\\[1cm] 

\flushright \Large Aleksandra Poręba \\ nr. indeksu 290514

\vfill 

\center{\today}


\pagebreak 

\end{center}

%----------------------------------------------------------------------------------------
%	SPIS TREŚCI
%----------------------------------------------------------------------------------------
\tableofcontents
\pagebreak

%----------------------------------------------------------------------------------------
%	ZAWARTOŚĆ
%----------------------------------------------------------------------------------------

\pagestyle{fancy}
\fancyhf{}

\rhead{\theauthor}
\lhead{\thetitle}
\cfoot{\thepage}

\section{Opis projektu}
W ramach projektu została stworzona aplikacja webowa, prezentująca automat komórkowy, jakim jest gra w życie.



Użytkownik może dowolnie dobierać zasady, przedstawione jako parametry STAY/BORN. Oznaczają one kolejno ilość żywych sąsiadów, aby komórka stała się żywa oraz ilość żywych sąsiadów, aby komórka pozostała żywa. Możliwy jest również wybór początkowej gęstości siatki oraz jej rozmiar.

Do stworzenia strony zostały użyte technologie HTML + JS + CSS.

\section{Ciekawe struktury}
\subsection{Life}
Jedną z ciekawych kombinacji jest 3/23, która jest regułami gry w życie według Conwaya. Po odpowiednio długim czasie mozna wyodrębnić różne ciekawe struktury: martwe, oscylujące, statki czy też działa.

%screen

\end{document}
